\section{Regressiotestaamisen merkitys}
 
\subsection{Luottamus ohjelmiston toimivuuteen}

Ohjelmistotuotannossa pyrit��n tuottamaan ohjelmisto, joka t�ytt�� sille asetetut vaatimukset. Jatkuva integraatio auttaa ohjelmistokehitt�ji� saavuttamaan vaatimukset ja tuottamaan enemm�n bisnesarvoa asiakkaille. Ohjelmistokehitt�jien on helpompi toteuttaa jatkuvaa integraatiota, kun he voivat luottaa ohjelmiston uusimman version toimivuuteen. Jatkuvassa integraatiossa ohjelmistokehitt�j� kommitoi ty�ns�, jonka j�lkeen erillisell� palvelulla buildataan ohjelmisto ja testataan, ett� se toimii.  


\newpage\phantom{blabla}
\newpage\phantom{blabla}

\subsection{Testikokonaisuuden valinta}

Regressiotestien tulee testata ohjelmiston oikeellisuutta ja paljastaa virheet. Siten testikokonaisuuden valintaan tulee kiinnitt�� erityist� huomiota. Er�s tapa on ajaa kaikki ohjelmiston testit ja testata kaikki testitapaukset. Se vie kuitenkin isossa projektissa liikaa aikaa, kun buildaamisen kesto on rajoitettu jatkuvassa integraatiossa kymmeneen minuuttiin. 

Testitapauksia voidaan yhdist�� tai niit� voi priorisoida. 

Ohjelmistokehitystiimi joutuu p��tt�m��n, mill� testikokonaisuudella ohjelmiston oikeellisuus voidaan varmistaa.


\newpage\phantom{blabla}
\newpage\phantom{blabla}

\subsection{Regressiotestit jatkuvassa muutoksessa}

Ohjelmiston kehitt�miseen kuuluu samalla uusien testien kehitt�minen. Ohjelmiston kasvaessa regressiotestien joukkoa joudutaan kasvattamaan tai ainakin muuttamaan.

\newpage\phantom{blabla}