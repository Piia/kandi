% !TEX root = kandi.tex


Jatkuva integraatio on ty�skentelytapa, jossa ohjelmistokehitt�j�t integroivat ty�ns� ohjelmiston p��versioon v�hint��n kerran p�iv�ss�. Sen tarkoituksena on toteuttaa ohjelmistotuotannon riskialttiit integraatiot mahdollisimman usein, jotta virheet huomattaisiin ja korjattaisiin ajoissa. N�in ohjelmistotuotannon projektit muuttuvat aikataulultaan ja kustannuksiltaan ennustettavammiksi. Integraatioon kuuluu ohjelmiston kokoaminen erillisell� jatkuvan integraation palvelimella. Kokoamisen yhteydess� ohjelmisto my�s testataan. Yleisin tapa t�ss� tilanteessa on regressiotestaaminen, joka keskittyy testaamaan muutoksen vaikutusta ohjelmistoon ja sen oikeellisuuteen. T�ss� tutkielmassa k�sitell��n jatkuvaa integraatiota ja regressiotestaamista. Jatkuvasta integraatiosta k�yd��n l�pi koontiversio (build) ja jatkuvan integraation hyv�t ja huonot puolet. Esimerkkitapauksena on hakukonej�tti Google. Lis�ksi tutkielmassa tarkastellaan regressiotestaamisen merkityst� jatkuvalle integraatiolle ja sen kustannuksille.