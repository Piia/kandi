% !TEX root = kandi.tex


Jatkuva integraatio on ty�skentelytapa, jossa ohjelmistokehitt�j�t inegroivat ty�ns� ohjelmiston p��versioon v�hint��n kerran p�iv�ss�. Sen tarkoituksena on toteuttaa ohjelmistotuotannon riskialttiit integraatiot mahdollisimman usein, jotta virheet huomataan ja korjataan ennen kuin ne kasvavat liian suuriksi. N�in ohjelmistotuotannon projektit ovat ennustettavampia. Integraatioon kuuluu ohjelmiston kokoaminen erillisell� jatkuvan integraation palvelimessa. Kokoamisen yhteydess� suoritetaan regressiotestaaminen, joka keskittyy testaamaan muutoksen vaikutusta ohjelmistoon ja sen oikeellisuuteen. Regressiotestit ovat testikokonaisuus, jonka muodostamiseen on erilaisia tekniikoita. Kokonaisuus keskittyy yleens� testaamaan muutoksen kannalta t�rkeit� ohjelmistonosia, sill� jatkuvan integraation tihe�tempoisen luonteen vuoksi kaikkia testej� ei ole mahdollista ajaa jatkuvan integraation palvelimella. Regressiotestaamisen kustannustehokkuus onkin jatkuvan integraation ydinkysymyksi�.