% !TEX root = kandi.tex


Jatkuva integraatio on ty�skentelytapa, jossa ohjelmistokehitt�j�t integroivat ty�ns� ohjelmiston p��versioon v�hint��n kerran p�iv�ss�. Sen tarkoituksena on toteuttaa ohjelmistotuotannon riskialttiit integraatiot mahdollisimman usein, jotta virheet huomataan ja korjataan ennen kuin ne kasvavat suuriksi. N�in ohjelmistotuotannon projektit muuttuvat ennustettavammiksi. Integraatioon kuuluu ohjelmiston kokoaminen erillisell� jatkuvan integraation palvelimella. Kokoamisen yhteydess� suoritetaan regressiotestaaminen, joka keskittyy testaamaan muutoksen vaikutusta ohjelmistoon ja sen oikeellisuuteen. 