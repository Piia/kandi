% !TEX root = kandi.tex
\section{Johdanto}

Jatkuva integraatio korvaa perinteisen ohjelmistokehityksen p��tt�v�n integraatiovaiheen. Se on ty�skentelytapa, jossa ohjelmistokehitt�j� integroi ty�t��n v�hint��n kerran p�iv�ss� ohjelmistokehityksen p��linjaan. Ohjelmistokehitt�j� aloittaa ty�skentelyns� hakemalla ohjelmiston uusimman version yhteisest� koodivarastosta ja kommitoi tehdyn ja testatun ty�n uusimmaksi versioksi saman p�iv�n aikana. Integraatiossa ohjelmisto kootaan ja ohjelmakoodin sis�lt�m�t automatisoidut testit ajetaan jatkuvaan integraatioon varatulla palvelimella. \cite{Fow06}

Regressiotestaamisella varmistetaan ohjelmiston oikeellisuus muutoksien j�lkeen. Regressiotestaaminen voidaan toteuttaa esimerkiksi valitsemalla kokemuksen perusteella sopiva joukko yksikk�-, integraatio- ja j�rjestelm�testej�. L�HDE Kaikkien mahdollisten testien suorittaminen ei ole kannattavaa, sill� se kuluttaa palvelinresursseja ja viivytt�� palautteen saamista. Ohjelmiston kokoamisen ja testien ajamisen pit�isi suosituksen mukaan kest�� alle kymmenen minuuttia. Testikokonaisuutta voi rajata priorisoimalla testitapauksia tai arvioimalla testien kriittisyytt�. Regressiotestej� kehitet��n ajan kuluessa varsinaisen ohjelmakoodin rinnalla. \cite{LIH17}

T�m�n kandidaatintutkielman luvussa yksi arvioidaan jatkuvaa integraatiota. Luvussa kaksi k�sitell��n regressiotestaamisen kustannustehokkuutta ja \par toteutustapoja. 

\clearpage