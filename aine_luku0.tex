% !TEX root = aine.tex

Aineessa tarkastellaan jatkuvaa integraatiota ja regressiotestaamista. Jatkuvaa integraatiota harjoittavat ohjelmistokehitt�j�t integroivat ty�ns� ohjelmiston p��versioon v�hint��n kerran p�iv�ss�. Jokainen muutos vaarantaa ohjelmiston toimivuuden, mutta jatkuvan integraation tarkoituksena on toteuttaa muutokset mahdollisimman usein ja korjata syntyneet virheet ennen kuin ne kasvavat liian suuriksi. Integraatioon kuuluu ohjelmiston kokoaminen ja regressiotestaaminen erillisell� jatkuvan integraation palvelimella. Regressiotestaaminen keskittyy testaamaan muutoksen vaikutusta ohjelmistoon ja sen oikeellisuuteen. Regressiotestit ovat testikokonaisuus, jonka muodostamiseen on erilaisia tekniikoita. Kokonaisuus keskittyy yleens� testaamaan muutoksen kannalta t�rkeit� ohjelmistonosia, sill� jatkuvan integraation tihe�tempoisen luonteen vuoksi kaikkia testej� ei ole mahdollista ajaa jatkuvan integraation palvelimella. 