% !TEX root = aine.tex
\section{Testikokoelman priorisointi ja rajaaminen}

\subsection{Rajalliset resurssit ja laatutekij�t}

Regressiotestaamisella pyrit��n ohjelmiston laadun lis��miseen ja ohjelmistokehityksen kustannusten v�henemiseen. Regressiotestaamista rajoittavat resurssien rajallisuus, kuten palvelinresurssit ja aikataulu. Miten l�yt�� pienin ohjelmaa testaava testikokonaisuus.

<Miten resurssit ja laatutekij�t ohjaavat testikokoelman priorisointia ja rajaamista, esittele lyhyesti eri tapoja.>
<software change impact analysis (= analyysi siit�, mihin osiin muutokset vaikuttavat) on ilmeisesti priorisointiin auttava v�line, mist� oli muistaakseni paperi siell� konferenssissa>
<On hyv� pit�� listaa testeist�, jotka eiv�t ole menneet l�pi. Listaa voidaan k�ytt�� my�hemmin herk�sti s�rkyv�n(fragile) koodin ennustamiseen.>

\clearpage

\subsection{Testidatan seuraaminen}

Testidatan liikkeist� voidaan regressiotestien suorituksen yhteydess� ker�t� metadataa. 

<Mit� reitti� testidata kulkee testin suorituksen aikana. Onko reitti odotettu. Metadataa voi my�s analysoida ja analyysin perusteella tehd� testikokonaisuutta koskevia valintoja>

<K�ytett�v�n testidatan valinta on t�rke�ss� roolissa, kuten testaamisessa yleens�kin. Onko se t�m�n aineen kannalta oleellista?>


\clearpage