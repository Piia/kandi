\section{Johdanto}

Jatkuva integraatio on ty�skentelytapa, jossa ohjelmistokehitt�j� integroi tuottamansa koodia jatkuvasti j�rjestelm��n. N�in ohjelmiston uusin versio on aina tarjolla ja se toimii. Ohjelmakoodi sis�lt�� automatisoidut testit, jotka ajetaan s��nn�llisesti testaukseen varatulla palvelimella. Se lis�� projektin l�pin�kyvyytt� ja tietoa ohjelmiston ajankohtaisesta kehitysvaiheesta. Jatkuvan integraation tarkoitus on korvata perinteisen ohjelmistokehityksen p��tt�v� integraatiovaihe. 

Regressiotestaaminen on nimitys toiminnoille, joilla testataan ohjelmiston oikeellisuutta muutosten j�lkeen. Sen voi toteuttaa esimerkiksi regressiotesteill�. Regressiotestit ovat testikokonaisuus, joka voi sis�lt�� yksikk�-, integraatio- ja j�rjestelm�testej�. 

Tutkielmassa tarkastellaan jatkuvan integraation ja regressiotestaamisen suhdetta. Miten ne vaikuttavat toisiinsa ja miten niit� voisi kehitt��.

Luvussa kaksi k�sitell��n regressiotestaamisen merkityst� jatkuvassa integraatiossa. Luvussa kolme k�yd��n l�pi regressiotestaamisen kehitt�mist� ja sen seurauksia jatkuvan integraation toteuttamiselle. Luvussa nelj� esitet��n johtop��t�kset.

\newpage\phantom{blabla}
\newpage\phantom{blabla}